cd\documentclass{article}

\usepackage{hyperref} % Package pour les liens hypertexte
\usepackage{graphicx} % Package pour inclure des images
\usepackage{tocloft} % Package pour personnaliser la table des matières
\usepackage[french]{babel}
\hypersetup{%
colorlinks=true,
linkcolor=blue,
urlcolor=blue
}
% Personnalisation de la table des matières
\renewcommand{\cftsecleader}{\cftdotfill{\cftdotsep}}
\setlength{\cftbeforesecskip}{5pt}

\title{Dossier au Format PDF}
\author{Votre Nom}

\begin{document}
\maketitle

\tableofcontents % Table des matières
\newpage

\section{Introduction}
 Dans le cadre de cet article de presse, je me lance dans une exploration passionnante de l'histoire des sciences, avec une attention particulière portée à l'astronomie. Mon choix s'est naturellement porté vers ce thème fascinant et sur lequel je m'efforce d'approfondir mes connaissances durant mon temps libre. Cette plongée dans l'histoire de l'astronomie et l'aspect prédictif des mathématiques constitue pour moi une aventure intellectuelle enrichissante. 

\section{Article 1} 
%Chercher article sur l'histoire des science, ses contributeurs, puis la fin qui se penche plus sur l'astronomie

Entre la fin du XVe siècle et le début du XVIIe siècle, une révolution scientifique fondamentale s'opère, remodelant radicalement les connaissances héritées du Moyen Âge. La méthode expérimentale, en opposition à l'enseignement aristotélicien préconisé par l'Église, devient prédominante grâce aux contributions marquantes de penseurs tels que Copernic, Kepler, Cardan, et surtout Galilée. La formulation d'une méthode scientifique distincte et d'un langage symbolique partagé par toutes les disciplines physiques marque une avancée significative. Cela permet à la pensée mathématique de se préciser, de se libérer de contraintes anciennes, et de se tourner résolument vers l'exploration du monde et de ses phénomènes.

L'astronomie, en particulier, connaît des avancées spectaculaires pendant cette période. Nicolas Copernic, au XVIe siècle, révolutionne la vision du cosmos en plaçant le Soleil au centre du système planétaire, écartant ainsi le modèle géocentrique ancien. Un siècle plus tard, Galileo Galilei pousse l'exploration du ciel à de nouveaux sommets grâce à l'utilisation d'une lunette astronomique révolutionnaire. Ses observations minutieuses de la Lune, de Jupiter et du Soleil ouvrent une fenêtre sur des réalités célestes insoupçonnées, bouleversant les conceptions établies depuis des siècles.

Par la suite les découvertes de Galilée ont été enrichie et développé par d'autres scientifiques tel que Einstein et sa théorie de la relativité général.

source: Les cahiers de Science et Vie. La naissance de la méthode scientifique

        La recherche. Histoire des sciences.

\label{sec:article1} % Étiquette pour créer un lien hypertexte
 % Texte de l'article 1

\section{Article 2} 
\subsection{Qu'est ce qu'un trou noir ?}
En 1796 Pierre-Simon Laplace spécule : "Imaginons une étoile, des millions de fois plus dense et plus compacte que le soleil. En vertu de la loi de la gravité, l'astre attirerait à lui tout ce qui l'approche, et rien ne pourrait en sortir, pas même la lumière.", Un siècle plus tard, en 1915, Albert Einstein, avec sa théorie de la relativité générale, a donné une base théorique solide à cette idée. Ses équations révolutionnaires démontrent que les trous noirs sont des objets d'une densité telle que l'intensité de leur champ gravitationnel empêche toute matière ou rayonnement de s'en échapper. Personne n'a encore réussi à remettre en question ces équations.

source : L'humanité. C'est le trou noir

Science et vie. Trou noir extrémal : le nouveau monstre cosmique

\label{sec:article2} % Étiquette pour créer un lien hypertexte

\section{Article 3}% 
%aspect prédicitif des maths. Parler du cliché réalisé et mettre une photo

Plus d'un siècle plus tard, un mercredi 10 avril à 15h17. Sur l'écran géant du siège de la commission européenne, les chercheurs observe un spéctacle extraordinaire: du gaz très chaud (plasma) et des fragments d'étoiles disloqués tourbillonnent en spirale autour d'un gouffre, avant de finalement y plonger, générant un sursaut brillant de lumière ultra violette. Ce phénomène, prédit par Albert Einstein dans ses équations sur la relativité générale, est enfin confirmé par cette observation. Il s'agit d'un trou noir d'une masse 6.5 milliards de fois supérieure à celle du soleil.

\includegraphics[width=300]{trou_noir.png}

source:L'humanité Le trou noir sort enfin de l'ombre
Europress

La recherche. Premiere image d'un trou noir



\label{sec:article3} 

\section{Article 4} %

Nous sommes en mesure de nous interroger sur l'existence d'autres prédictions au sein de la relativité générale d'Einstein qui n'ont pas encore trouvé de confirmation expérimentale.
\subsection{Les trou de ver}
Les trous de ver, objets hypothétiques reliant deux régions distinctes de l'espace-temps, demeurent encore à être observés expérimentalement. Théorisés il y a près de 100 ans par Albert Einstein et Nathan Rosen en 1935, ces tunnels à travers le tissu de l'espace-temps seraient des raccourcis entre le centre de deux trous noirs, conformément à la théorie générale de la relativité d'Einstein, qui décrit la gravité comme une courbure de l'espace-temps. 
\subsection{Les trous blancs}
Les trous blancs, également des solutions des équations de la relativité générale, partagent la même base théorique que les trous noirs. En effet, les trous noirs pourraient donner naissance aux trous blancs, des structures où la matière ne peut pas tomber mais seulement remonter et ressortir. On pourrait envisager un trou blanc comme un trou noir où la flèche du temps serait inversée.



Source : Science et vie. Des physiciens simulent un mini trou de ver à l’aide d’un ordinateur quantique

Les Echos. Et si l'univers était criblé de trous blancs ?

\label{sec:article4} 

\section{Article 5}

D'autres scientifique dans la lignée d'Einstein on a travers leurs équations fait des prédictions. 

u milieu du 19e siècle, des astronomes ont remarqué des irrégularités dans l'orbite d'Uranus qui ne pouvaient pas être expliquées uniquement par la gravité des planètes connues à l'époque. Urbain Le Verrier en France et John Couch Adams en Angleterre ont indépendamment utilisé les lois de la gravité de Newton pour prédire l'existence et la position d'une planète inconnue qui pourrait expliquer ces anomalies.

En 1846, Johann Gottfried Galle a observé Neptune près de la position prédite par Le Verrier et Adams, confirmant ainsi l'existence de la planète et la validité de leurs calculs.

Cette découverte a été un triomphe pour la théorie de la gravitation et a montré comment les calculs mathématiques pouvaient conduire à la découverte d'objets célestes avant même de les observer directement.

source : La recherche Histoire des sciences : TEMPÊTE AUTOUR DE LA DÉCOUVERTE DE NEPTUNE
\section{Conclusion}

\end{document}
